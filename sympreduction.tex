The standard geometric quantization procedure works on symplectic manifolds, a significant subclass of which --- including all unconstrained physical phase spaces --- is comprised of cotangent bundles. But when do we obtain a symplectic manifold that is \emph{not} equivalent to a cotangent bundle?

In practice, we obtain a symplectic manifold that is inequivalent to a cotangent bundle when we have a symmetry acting on phase space (viewed as a Lie group action) and proceed to obtain the ``reduction'' of the space via the action of the symmetry. In general, the process is known as \emph{symplectic reduction}; this section will be devoted to this reduction procedure.

Generally speaking, symplectic reduction assumes a symplectic manifold $M$ equipped with a Lie group action $G \circlearrowleft M$ on $M$, producing a symplectic manifold $M // G$ --- known as the \emph{reduced space} --- in the following manner:

\begin{itemize}
\item Let $\mathfrak{g}$ be the Lie algebra associated to the Lie group $G$, i.e. the exponential map takes $\mathfrak{g}$ to $G$. If the action $G \circlearrowleft M$ preserves the symplectic form on $M$, then we obtain the following homomorphisms of Lie algebras:
$$
(\cdot)^M : \mathfrak{g} \to \Gamma(M,TM), X \mapsto X^M
$$
constructed by composing the group action and the exponential map, and  %CLARIFY!!!!!! --insert commutative diagram here to show how the map works
$$
J_{(\cdot)}: \mathfrak{g} \to C^\infty(M), X \mapsto J_X \mbox{ s.t. } X^M \cong \xi_{J_X},
$$
where $\xi_Y$ is the Hamiltonian vector field of $Y$.

\item The functions $J_X$ may be assembled into a \emph{moment map}\footnote{Also written as ``momentum mapping''.} $J: M \to \mathfrak{g}^*$ for the action:
$$
\langle J(p) , X \rangle = J_X(p).
$$
Here, $\mathfrak{g}^*$ refers to the vector space dual of the Lie algebra $\mathfrak{g}$.

\item The reduced space $M^0$ is then defined via $M^0 = J^{-1}(0) / G$, the preimage of $0 \in \mathfrak{g}^0$ through $J$, modulo $G$ Hence $M^0$ is a subspace of $M / G$, the orbit space of the action $G \circlearrowleft M$. Generally, we cannot guarantee that $M^0$ is a smooth manifold unless certain topological assumptions are made on the Lie group and its action, which we review below; however, if those conditions hold and $M^0$ is a smooth manifold, it also inherits a natural symplectic form.
\end{itemize}

In this chapter, we define and discuss the notion of symplectic reduction, with a particular focus on the symplectic reduction process of Marsden and Weinstein developed in \cite{marsdenweinstein}.















\section{Lie Group Actions on Symplectic and Poisson Manifolds}
In general, actions of Lie groups on smooth manifolds allow us a secondary perspective towards understanding complicated manifold geometries. Lie group actions form ``layerings'' in the sense that we may study $g \cdot M$ for each $g in G$; at the same time, we can also look at the \emph{orbit-types} of $G$ %%%what does this mean???? http://en.wikipedia.org/wiki/Topologically_stratified_space

More formally, suppose we had an action of a Lie group on a smooth manifold given by
$$
\sigma: G \times M \to M, (g,p) \mapsto g \cdot p;
$$
for a point $p \in M$, consider the orbit map $\sigma_p: G \to M$ given by $\sigma_p(g) = g \cdot p$. Then $\sigma_p$ is a differentiable map, and its differential at $e \in G$ is %%%%%finish notes in "lie groups acting on symplectic manifolds pt. I"

\subsection{On Symplectic Manifolds}
Fix a connected Lie group $G$ with an action $G \circlearrowleft M$ on a symplectic manifold $(M,\omega)$. Suppose that the action $G \circlearrowleft M$ leaves $\omega$ fixed, i.e. $G$ acts on $M$ through symplectomorphisms $M \to M$. Let $\mathfrak{g}$ be the Lie algebra associated to $G$, i.e. $\exp G = \mathfrak{g}$. Then for $X \in \mathfrak{g}$, we have that
$$
0 = X \cdot \omega = \d \iota(\xi_X)\omega + \iota(\xi_X)\d \omega = \d \iota(\xi_X)\omega \mbox{ (~ closed one-form),}
$$
where $\iota$ is the map that [...FIGURE THIS PHRASING OUT], $\d$ is the exterior derivative, and $\xi_X$ is the vector field on $M$ associated to $X$. The first equality follows from the definition of a symplectomorphism; the second from the definition of the Lie algebra action; and the third from the fact that $\omega$ is closed. %FIX THIS

The above chain of equalities leads to two useful definitions. Call a vector field $\xi \in \Chi(M)$ a \emph{symplectic vector field} if
$$
\d \iota(\xi)\omega \mbox{ is closed,}
$$
and call it a \emph{Hamiltonian vector field} if
$$
\xi^\flat \mbox{ is exact,}
$$
i.e. if there exists a function $\phi_\xi \in C^\infty(M)$ such that $\xi^\flat + \d \phi_\xi = 0$\footnote{Observe that $\phi_\xi$ is in general \emph{not} unique.}.

Define also the corresponding spaces $\mathfrak{sym}(M)$ and $\mathfrak{ham}(M)$ consisting of the above vector fields:
\begin{itemize}
\item Define $\mathfrak{sym}(M)$ to be the space of symplectic vector fields on $M$; we also note that
$$
\mathfrak{sym}(M) = \sharp(\Omega^1_{closed}(M))
$$
where $\sharp$ is the \emph{musical isomorphism} taking a differential 1-form to its corresponding vector field.

\item Define $\mathfrak{ham}(M)$ to be the space of hamiltonian vector fields on $M$; we note that
$$
\mathfrak{ham}(M) = \sharp(\Omega^1_{exact}(M)).
$$
 \end{itemize}

With the above definitions in hand, we can now define the notions of \emph{Hamiltonian} and \emph{Poisson} actions of Lie groups:
\begin{defn}
Let $G \circlearrowleft M$ be a Lie group action, and let $\mathfrak{g} = \Lie(G)$ be the Lie algebra associated to $G$. We say that $G \circlearrowleft M$ is \emph{Hamiltonian} if, for all $X \in \mathfrak{g}$, we can find a function $\phi_X \in C^\infty(M)$  such that
$$
\xi^\flat_X + \d \phi_X = 0,
$$
in which case we obtain an associated map $\mathfrak{g} \to C^\infty(M)$ given by %%%%

Further, consider the Poisson algebra on $C^\infty(M)$ given by the Poisson bracket
$$
\{f,g\} := \omega(\xi_f, \xi_g),
$$
where $\xi_f$ (resp., $\xi_g$) is the hamiltonian vector field asociated to $f$ (resp., $g$), i.e. $\xi^\flat_f + \d f = 0$ ($\xi^\flat_g + \d g = 0$). This gives us a Lie algebra structure on $C^\infty(M)$, and a Hamiltonian action $G \circlearrowleft M$ is said to be \emph{Poisson} if there exists a homomorphism of Lie algebras $\mathfrak{g} \to C^\infty(M), X \mapsto \phi_X$ such that
$$
\xi^\flat_X + \d \phi_X = 0 \mbox{ and } \phi_{[X,Y]} = \{\phi_X, \phi_Y\},
$$
for all $X,Y \in \mathfrak{g}$.
\end{defn}

Intuitively, %%%%%PARSE DEFINITION & PROVIDE INTUITION/MOTIVATION HERE

\subsection{On Poisson Manifolds}
Fix a Poisson manifold $P$, i.e. a smooth manifold with a Lie bracket $\{\cdot,\cdot\}$ on its space of smooth functions $C^\infty(P)$ such that the \emph{Leibniz rule} holds:
$$
\forall f,g,h \in C^\infty(M), \{fg,h\} = f\{g,h\} + g\{f,h\}.
$$
Equivalently, we may say that Poisson manifolds are smooth manifolds such that $C^\infty(M)$ is equipped with a Lie algebra structure such that the %%%quote wiki:
%http://en.wikipedia.org/wiki/Poisson_manifold

\begin{example}
As an example of a Poisson bracket, consider the \emph{Hamiltonian vector field} $\xi_f$ of a function $f$, i.e. the unique vector field satisfying
$$
% define hamiltonian vector fields here
$$
The Lie bracket of two vector fields is defined via $[X,Y] = XY - YX$, viewing the vector fields as derivations. It is not difficult to verify that the following map is a Poisson bracket on $C^\infty(M)$:
$$
\{\cdot, \cdot\}: C^\infty(M) \times C^\infty(M) \to C^\infty(M), \{f,g\} := [\xi_f,\xi_g].
$$
\end{example}

For a Lie group $G$, a \emph{Lie group action} on a Poisson manifold $P$ 
% action of a lie group on P via its lie algebra action

\subsection{Closing Remarks on Symplectic and Poisson Manifolds}
There are a few remarks to make here regarding the connection between symplectic and Poisson manifolds. The most obvious is that every symplectic manifold possesses a Poisson manifold if one takes the Poisson bracket to be given by
$$
\{f,g\} = \omega(\xi_f,\xi_g),
$$
where as above $\xi_f$ (resp., $\xi_g$) is the Hamiltonian vector field of $f$ (resp., $g$).

Something more can be said here --- symplectic manifolds are those special Poisson manifolds in which the Hamiltonian vector fields \emph{exhaust} the tangent bundle:
\begin{thrm}
%prove the above here & make rigorous
\end{thrm}

Finally, we would like a way of measuring the difference between symplectic actions and Poisson actions of Lie groups, i.e. a way to quantify the obstructions that would prevent a symplectic action from becoming a Poisson action. This is possible through a combination of the \emph{de Rham} and \emph{Chevalley-Eilenberg}\footnote{Also known as \emph{Lie algebra} cohomology.} cohomology theories --- and more generally, via \emph{equivariant cohomology}.
% short discussion on deRham+chevalley-eilenberg / equivariant cohomology here.
















\section{Marsden-Weinstein Reduction}
To define the Marsden-Weinstein reduction process, one must first introduce the moment map taking a manifold to the dual of the Lie algebra acting on it.

%COVER NOTES IN "MOMENT MAPS & HAMILTONIAN GROUP ACTIONS" HERE -- SEE NOTEBOOK

\subsection{Definition}
% 1)
% 2)
% 3)
\subsection{Properties}
% 1) __SLICE THM__: when is M^0 a manifold? ---> when we impose (i) 0 is a __regular value__ of J; && (ii) the G-action is __proper__ and __free__ on J^{-1}(0).
% 2) When the above holds, M^0 is a manifold with a unique symplectic form \omega^0 satisfying:
$$
\iota^*\omega = \pi^*\omega^0,
$$
where the maps $\iota$ and $\pi$ are inclusion and projection maps, i.e.
$$
M \overset{\iota}\hookleftarrow J^{-1}(0) \overset{\pi}\to M^0.
$$















\section{Coisotropic Reduction}
Aside from the notion of Marsden-Weinstein reduction, there is also \emph{coisotropic} reduction, in which
%%%%%%%%%%%%%%%% COVER COISOTROPIC REDUCTION HERE






\section{Historical Notes and References}
The notion of a Lie group action on a manifold has existed since %%%%

In general, the orbit space $M / G$ of a $G$-manifold, when $G$ is a general Lie group, fails to be a manifold unless we impose requirements on $G$. For a study of how to deal with orbifold orbit spaces, see \cite{FIGURE SOMETHING OUT LOOK IT UP}.
% ^why? slice theorem: http://en.wikipedia.org/wiki/Slice_theorem_%28differential_geometry%29

The genesis of Lie algebra cohomology is in the 1948 paper of Chevalley and Eilenberg, ``Cohomology Theory of Lie Groups and Lie Algebras'' \cite{chevalleyeilenberg}; a modern introduction may be found in the seventh chapter of Hilton & Stammbach's ``A Course in Homological Algebra'' \cite{HiltonStammbach}.

For a cursory introduction to equivariant cohomology, there is the short expository by L. Tu \cite{tuequivariant};
%http://www.ams.org/notices/201103/rtx110300423p.pdf
deeper discussions may be found in the first chapter of Guillemin and Sternberg's ``Supersymmetry and Equivariant de Rham Theory''
% http://www.springer.com/gp/book/9783540647973
and % look through http://ncatlab.org/nlab/show/equivariant+cohomology#references