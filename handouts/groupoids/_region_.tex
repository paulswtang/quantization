\message{ !name(sympgroupoids.tex)}%-----PREAMBLE-----
\documentclass{tufte-handout}

%use tikz-cd for comm diagrams:
%\usepackage{tikz-cd}
%use amsthm for thm-like environments:
\usepackage{amsthm}
%use thmtools to style new thm envns:
\usepackage{thmtools}
% amsfonts for fraktur lettering:
\usepackage{amsfont}

%define thrm environment
\newtheorem{thrm}{Theorem}

%define example environment
\newtheorem{example}{Example}

%define proof-sketch environment
\renewcommand{\qedsymbol}{$\triangle$} %use triangle instead of qed square

%define definition environment
\newtheorem{defn}{Definition}

%define shorthand for exterior derivative:
\def\d{\mathrm{d}}

\title{Symplectic Groupoids}

%-----DOCUMENT BODY-----
\begin{document}

\message{ !name(sympgroupoids.tex) !offset(-3) }


In this chapter we will present the basic details of \emph{symplectic groupoids}, a mutual category-theoretic generalization of symplectic and Poisson manifolds into the groupoid setting; if groups naturally arise out of symmetry, groupoids naturally arise out of ``internal'' symmetries with respect to a bundle-type structure, i.e. geometric objects with algebro-geometric objects attached at each point. We proceed in a gradual build-up, first defining groupoids and then making our way through Lie groupoids/algebroids and then encountering symplectic groupoids.

We begin with the definition of a groupoid by presenting it both in the classical sense and in the category-theoretic sense; then comes a presentation of the $*$-oidification\footnote{(The ``$*$-oidification'' process can be thought of as analogous to the process of passing from the fundamental group on a fixed basepoint to the fundamental groupoid, which removes the dependency on a chosen base point.)} of Lie groups and Lie algebras; and we end with a discussion of symplectic groupoids.

%=============================================================
\section{Groupoids}
There are two equivalent definitions of the notion of a groupoid --- one follows a traditional set-theoretic definition and the other defines them as a certain type of category. First, the traditional definition:

\begin{defn}
Let $\Gamma$ be a set with subset $\Gamma_0$ of \emph{identity elements} along with projections $\alpha, \beta: \Gamma \to \Gamma_0$. Let there be a multiplication operation $(x,y) \mapsto x \cdot y$ on the set
$$
\{(x,y) \in \Gamma \times \Gamma : \beta(x) = \alpha(y)\},
$$
and let there be an inverse operation $(\cdot)^{-1}: \Gamma \to \Gamma$ such that the product and the inverse satisfy the usual group axioms on their domain.
\end{defn}
Essentially, a groupoid is defined as a set with a partially defined product operation --- we can only multiply two elements if they map to the same identity. (Note also that we can enforce commutativity when we specify $\alpha = \beta$.)

The category-theoretic definition is easier to state:
\begin{defn}
A \emph{groupoid} is a category in which all morphisms are invertible.
\end{defn}
The proof of equivalence (with the caveat that the category is \emph{small}) is not difficult, but is tiresome and mainly syntactic manipulation. One may consult any exposition of groupoids for the proof (in fact, even Wikipedia suffices).

We have the following examples of groupoids occurring in mathematics:
\begin{itemize}
\item Every group may be thought of as a single-object category with invertible morphisms; hence the notion of groupoid is a proper generalization of the notion of a group.

\item \emph{Fundamental Groupoids.} Recall that path-connected topological spaces have fundamental groups that are invariant with respect to choices of base points; the fundamental groupoid $\pi_1(X)$ for a general topological space $X$, then, is defined as the category with points of $X$ as objects and paths (up to homotopy equivalence) as arrows. Concatenation of paths is used as composition of arrows. The standard fundamental group $\pi_1(X,x)$ at a point $x \in X$ can then be seen as the vertex group (at $x$) of the fundamental groupoid.

\item \emph{Action Groupoids.} Suppose we have a group $G$ acting on a set $S$; then the associated action groupoid $S//G$ is given by the category with elements of $S$ as objects and group elements $g \cdot s_1 = s_2$ as arrows from $s_1$ to $s_2$. Compositions are done in the natural way, i.e. by applying one group element after another using the action. We note that any two objects $s_1,s_2$ in the same orbit are isomorphic (in the category-theoretic sense).
\end{itemize}

%=============================================================
\section{Lie Groupoids and Lie Algebroids}
Lie groupoids may be thought of as groupoids in the setting of smooth manifolds, or as ``many-object'' versions of Lie groups (in the same way that groupoids are ``many-object'' versions of groups). More specifically,

\begin{defn}
A \emph{Lie groupoid} is a groupoid in which the inclusion maps $\alpha, \beta : \Gamma \to \Gamma_0$ are submersions, the multiplication and inversion operations are smooth, and $\Gamma, \Gamma_0$ are both smooth manifolds.

Equivalently, a Lie groupoid is a groupoid internal to the category of smooth manifolds; in other words, a Lie groupoid $\Gamma$ over a base manifold $\Gamma_0$ has objects as elements of $\Gamma_0$, hom-sets $hom_\Gamma(p,q)$ a manifold for $p,q \in \Gamma_0$, with all arrows invertible and source/target maps $\alpha,\beta$ both submersive.

We have the following commutative diagram:

%INSERT COMM DIAGRAM #1.
\end{defn}

We also have the notion of Lie algebroid:
\begin{defn}
A \emph{Lie algebroid} is
\begin{itemize}
\item a vector bundle $\pi: E \to M$, along with
\item a Lie bracket $[\cdot,\cdot]$ on the space $\Gamma(E)$ of sections of $E$, and
\item an \emph{anchor}, i.e. a vector bundle morphism $E \to TM$ from the bundle to the tangent bundle of the base manifold.
\end{itemize}
\end{defn}
To put it succinctly, a Lie algebroid is a vector bundle with a Lie algebra attached to every point of the base manifold.

\begin{example}[Tangent Lie Algebroid]
Let $X$ be a smooth manifold. The \emph{tangent Lie algebroid} of $X$, denoted $TX$, is the Lie algebroid corresponding to the fundamental groupoid $\Pi_1 X$ of $X$.

Seen another way, the tangent Lie algebroid of $X$ is made from the collection of tangent spaces $T_p X$ (for $p \in X$) after ``forgetting'' the base point $p$ and going through $*$-oid-ification.% consider expanding
\end{example}

The functorial association of a Lie algebra to Lie groups extends to a functorial association of a Lie algebroid to a Lie groupoid:

\begin{thrm} %lie algebroid associated to a lie groupoid:
% from http://empg.maths.ed.ac.uk/Activities/GCY/Racaniere.pdf
Let $G \rightrightarrows M$ be a Lie groupoid (with overall manifold $G$, base manifold $M$, and source/target maps $s,t: G \to M$). Then there exists a unique Lie algebroid $A \to M$ given by the following:
\begin{itemize}
\item The bundle $A$ is given by the pullback of $T_s : T_sG \to TM$ across the embedding $\iota: M \to G$; in other words, the restriction of the vector bundle over $G$ induced by the target map $t$ to $M$. Hence, we have the commutative diagram

%INSERT COMM DIAGRAM #5.

To rephrase this, the kernel of the target map $t$ is a vector bundle $T^t G \to G$; $A$ is then defined as the restriction of $T^t G$ to $M$. Equivalently,
$$
A := \iota^* T^t G.
$$

\item The bracket $[\cdot,\cdot]$ is inherited from the Lie bracket on vector fields on $G$, since each section $\sigma: M \to A$ can be identified with a unique vector field $X_\sigma: G \to TG$ via
$$
X_\sigma (x) = L_x \sigma(s(x)),
$$
where $L_x$ is the isomorphism
$$
L_x : T^t_y G \to T^t_{x \cdot y} G.
$$

\item The anchor map is given by the differential $ds$ of the source map $s$.
\end{itemize}

The association is functorial and the following diagram is commutative:

%INSERT COMM DIAGRAM #2.

\end{thrm}

The three theorems of Lie (see appendix) generalize to the theory of Lie groupoids and algebroids with partial success; we present them side by side here.

\begin{thrm}

%extensions of lie's 3 thrms to the *-oid setting
Recall that we have a functor from the category of Lie groups to Lie algebras
$$
\mathrm{Lie}: \mathrm{LieGrp} \to \mathrm{LieAlg}
$$
given by
$$
\mathrm{Lie}(G) := \mathfrak{g} \cong T_e G
$$
and, for $f: G \to H$ a morphism of Lie groups,
$$
\mathrm{Lie}(f): \mathfrak{g} \to \mathfrak{h}
$$
is defined as the unique map making the following diagram commute:

% INSERT COMM DIAGRAM OF LIE FUNCTOR HERE.

Further, we have a \emph{differentiation} functor
$$
\mathrm{diff}: \mathrm{LieGroupoid} \to \mathrm{LieAlgebroid}
$$
given by the construction in the previous section.


\begin{itemize}
\item The first Lie theorem is a local one and it is not readily apparent how to translate the concepts to general manifolds.

\item The second Lie theorem states that for $G,H$ Lie groups with $\mathfrak{g} = Lie(G), \mathfrak{h} = Lie(H)$ and $\pi_1(G) = 0$, we have that a Lie algebra morphism $f: \mathfrak{g} \to \mathfrak{h}$ has a lift $F: G \to H$ such that $Lie(F) = f$. This generalizes in a straightforward manner to groupoids and algebroids; we call a Lie groupoid \emph{(source-)simply connected} if the fibers of the source map are simply connected for all $x \in \Gamma_0$ the base manifold. Then the corresponding statement is true when we use this definition of simple-connectedness on Lie groupoids.

\item The third Lie theorem\footnote{Also known as the \emph{Cartan-Lie} theorem, for Elie Cartan.}, which states that the $Lie$ functor is all finite-dimensional real Lie algebras $\mathfrak{g}$ are the Lie algebra of some real Lie group $G$ (i.e. $\exists G$ such that $\mathfrak{g} \cong Lie(G)$), fails to generalize to the *-oid setting in a spectacular way --- integrating a Lie algebroid often fails to give a manifold as it may introduce singularities. % consider elaborating (see racaniere section 5)

\end{itemize}
\end{thrm}

The spectacular failure of Lie's third theorem in the algebroid setting hints at the fact that the structure theory of Lie groupoids/algebroids need some fine tuning; one idea is to ``stackify'' the structures by considering ``stacky'' groupoids, as proposed by Tseng-Zhu in ``Integrating Lie Algebroids via Stacks'' \cite{tsengzhu}.

We will not proceed further with the discussion of the theory of Lie groupoids. A detailed introduction to Lie groupoids and algebroids may be found in Mackenzie, ``General Theory of Lie Groupoids and Lie Algebroids'' \cite{mackenzie}.

%=============================================================
\section{Symplectic Groupoids}
Symplectic groupoids are Lie groupoids with a symplectic structure on the manifold of morphisms such that the internal base manifold is Poisson and the groupoid product respects the symplectic form (in a suitable sense).

More formally, we have the following definition:
\begin{defn}
A \emph{symplectic groupoid} is a Lie groupoid $\Gamma \rightrightarrows M$ with $\Gamma, M$ both smooth manifolds and source/target maps $s,t: \Gamma \to M$ submersive; further, there is a symplectic structure on $\Gamma$ given by a form $\omega$ such that the submanifold
$$
L = \{(x,y,z) | xy = z \} \subset (\Gamma,-\omega) \times (\Gamma,-\omega) \times (\Gamma,\omega)
$$
is a Lagrangian submanifold.

In other words, a Lie groupoid is \emph{symplectic} if the space of morphisms $\Gamma$ is symplectic with a multiplicative symplectic form.
\end{defn} %see racaniere, section 6 --- lemma 6.1.

%results about symplectic groupoids:
The notion of a symplectic groupoid is remarkable for a few reasons. Recalling that symplectic manifolds may be viewed as Poisson manifolds (by noting that the bracket given by the symplectic form\footnote{} defines a natural Poisson structure), we note that %%%%%
\begin{thrm}
Let $\Gamma \rightrightarrow M$ be a symplectic manifold. Then %MANIFOLD OF IDENTITY ELEMENTS IS A POISSON MANIFOLD; WEINSTEIN87
\end{thrm}

Further, we have a functorial correspondence between symplectic groupoids and Poisson manifolds, in the style of the Lie group-Lie algebra correspondence:
\begin{thrm}
Consider the categories $\mathrm{SympGrpd}$ and $\mathrm{PoisMfd}$ of symplectic groupoids and Poisson manifolds, respectively. Then there is a functor %FUNCTORIAL CORRESP. BTWN SYMP. GRPDS. & POISSON MFDS., EXTENDS LIE GRPS. & LIE ALGS. FUNCTOR; INSERT DIAGRAM 3
\end{thrm}
The proof of this statement is outside the scope of this writeup; a presentation may be found in \textbf{[weinstein87]}.

%=============================================================
\section{Historical Notes and References}
The first definition of a symplectic groupoid comes from Weinstein, ``Symplectic Groupoids and Poisson Manifolds'' \cite{weinstein87} as well as Karasev \cite{} and Zakrzewski \cite{}; all were motivated by the problem of understanding quantization. The notion of groupoids as categories generalizing groups came earlier \cite{SOMETHING HERE}. A good reference for this topic is \cite{}. %see http://ncatlab.org/nlab/show/symplectic+groupoid#references

Symplectic groupoids were introduced in \cite{weinstein87}; %history of development

A survey on groupoids, including several examples, may be found in \cite{weinstein1996} %"groupoids: unifying external and internal"

A useful reference on Lie groupoids and Lie algebroids in a geometric context may be found in \cite{mackenzie1987}
%kirill mackenzie: lie groupoids and lie algebroids in differential geometry

% tangent lie algebroid: http://iopscience.iop.org/article/10.1088/0305-4470/27/13/026/pdf;jsessionid=BC34ABF19D2678BE3DF781F316CFE10F.c3.iopscience.cld.iop.org

\end{document}
\message{ !name(sympgroupoids.tex) !offset(-215) }
