\message{ !name(introgeomquant.tex)}%-----PREAMBLE-----
\documentclass{tufte-handout}

%use tikz-cd for comm diagrams:
%\usepackage{tikz-cd}
%use amsthm for thm-like environments:
\usepackage{amsthm}
%use thmtools to style new thm envns:
\usepackage{thmtools}

%define example environment
\newtheorem{example}{Example}

%define proof-sketch environment
\renewcommand{\qedsymbol}{$\triangle$}

%define definition environment
\newtheorem{defn}{Definition}

%define shorthand for exterior derivative:
\def{\d}{\mathrm{d}}

\title{An Overview to the Geometric Theory of Quantization}


%-----DOCUMENT BODY-----
\begin{document}

\message{ !name(introgeomquant.tex) !offset(174) }
\subsection{Symmetries on a Hilbert Space}
%symmetries
We also have the following definition of a \emph{symmetry} on a Hilbert space, due to E. Wigner:
\begin{defn}
Let $\mathcal{H}$ be a complex Hilbert space; then a bijective map $U: \mathcal{H} \to \mathcal{H}$ is called a (quantum) \emph{symmetry} iff the ``transition probabilities'' are preserved under $U$, i.e.
$$
\forall \psi,\phi \in \mathcal{H}, \frac{|\langle U(\psi),U(\phi)\rangle|^2}{|| U(\phi) ||^2 \cdot || U(\psi) ||^2} = \frac{|\langle \psi,\phi\rangle|^2}{|| \phi ||^2 \cdot || \psi ||^2}.
$$
\end{defn}
Intuitively, symmetries may be thought of as unitary or anti-unitary operators:

%results on symmetry: proposition, all symmetries come from unitaries and anti-unitaries; show how this justifies the definition of the category of hilbert spaces
\begin{thrm}
%%%%%%%%%%%%%%%%%%%%%%%%%%%%%%proposition: all symmetries come from unitaries and anti-unitaries
\end{thrm}

Further, we have the following properties regarding the category of Hilbert spaces (as objects) and unitary operators (as arrows):
\begin{thrm}
%%%%%%%%%%%%%%%%%%%%%%%%%%%%%% category of hilbert spaces and unitaries; maybe not as obvious as one might think??? see following links:
%http://math.ucr.edu/home/baez/quantum/node4.html
%http://mathoverflow.net/questions/34058/does-the-category-of-hilbert-spaces-possess-a-product
\end{thrm}

We will not mention much more about the theory of Hilbert spaces here; for further details, consult the appendix and the references.

%----- ----- ----- ----- -----

\message{ !name(introgeomquant.tex) !offset(236) }

\end{document}
