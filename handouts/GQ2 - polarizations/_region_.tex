\message{ !name(polarizations.tex)}%-----PREAMBLE-----
\documentclass{tufte-handout}

%use tikz-cd for comm diagrams:
%\usepackage{tikz-cd}
%use amsthm for thm-like environments:
\usepackage{amsthm}
%use thmtools to style new thm envns:
\usepackage{thmtools}

%define thrm environment
\newtheorem{thrm}{Theorem}

%define example environment
\newtheorem{example}{Example}

%define proof-sketch environment
\renewcommand{\qedsymbol}{$\triangle$}

%define definition environment
\newtheorem{defn}{Definition}

%define shorthand for exterior derivative:
\def\d{\mathrm{d}}

\title{Polarizations and Quantization}


%-----DOCUMENT BODY-----
\begin{document}

\message{ !name(polarizations.tex) !offset(1) }
\section{Definition and Basic Results}

We will hit the ground running with the definitions and proceed to justify it shortly after:

\begin{defn}
Let $(M^{2n},\omega)$ be a symplectic manifold. A (complex) distribution $P$ on $M$, i.e. a sub-bundle
$$
P \hookrightarrow T^{\mathbf{C}}M \equiv TM \otimes_{\mathbf{R}} \mathbf{C}
$$
of the (complexified) tangent bundle, is said to be a \emph{polarization} of $M$ if all of the following are hold:
\begin{itemize}
\item $P$ is \emph{Lagrangian}, i.e. it is an \emph{isotropic} distribution of dimension $n$; recall that a distribution is said to be isotropic if the symplectic form $\omega_x$ on each tangent space $T_xM$ vanishes on the subspace $P_x \hookrightarrow T_xM$ defined by the distribution, and that the maximal dimension of an isotropic distribution is half the dimension of the manifold.

\item $P$ is \emph{involutive}, i.e. for $X,Y$ vector fields in $P$, the commutator $[X,Y]$ is a vector field in $P$. In the literature, this is usually written in shorthand as $[P,P] \subset P$.

\item For all points $m \in M$, we have that $\mathrm{dim}(P_m \cap \overline{P}_m \cap TM)$ is constant, where the overline refers to complex conjugation.

\end{itemize}
\end{defn}

Having defined the meaning of a polarization, we can now define a way of polarizing the prequantum Hilbert space and polarizing the operators obtained through prequantization.
\begin{defn}
Given a symplectic manifold $(M^{2n},\Omega)$ with prequantum line bundle $(L \to M,\nabla)$ and prequantum Hilbert space\footnote{(Or more accurately, the metric completion of this space.)}
$$
\mathcal{H}^{preq} := \{\psi \in \Gamma(L) | \langle\psi,\psi\rangle < \infty \},
$$
we define the polarized Hilbert space $\mathcal{H}^{P}$ for $P$ a choice of polarization on $M$ as the Hilbert space of sections in $\mathcal{H}^{preq}$ that \emph{respect the polarization}, i.e.
$$
\nabla_X \psi = 0 \qquad \forall X \in \mathcal{X}(P),
$$
where $X$ is a vector field generated by $P$, i.e. the space of sections that remain fixed by parallel transport along any $X$ in the polarization.

Further, we say that an observable $f \in C^\infty(M)$ is \emph{polarizable}, or \emph{``respects the polarization''}, under the polarization $P$ iff $X_f$ the associated Hamiltonian vector field satisfies\footnote{Here, this is shorthand for the fact that $X_f$ keeps $P$ fixed.}
$$
[X_f,P] = P.
$$
When $f$ is polarizable, its corresponding \emph{polarized operator} $O_f: \mathcal{H}^{P} \to \mathcal{H}^{P}$ is the same\footnote{In other words, the primary purpose of a choice of polarization is to restrict the class of possible classical observables that may be meaningfully quantized --- it does not change the definition of the quantization of a classical observable.}
 as the prequantum operator $O^{preq}_f$.

\end{defn}

We remark that the condition placed upon the smooth functions on $M$ is necessary as not all observables respect the polarization. For instance, given a polarization $P$ on $M$ such that $\overline{P} = P$, we can find coordinates $x_1,\ldots,x_n,y_1,\ldots,y_n$ such that
$$
P = span(\{\frac{\partial}{\partial x_i}_{i=1}^n\});
$$
then one can verify that the oberservables $y_1(p),\ldots,y_n(p) \in C^\infty(M)$ fail to respect the polarization.
% see http://www.math.toronto.edu/karshon/grad/2006-07/polarizations.pdf for reference


\subsection{Justification and Intuition.}
We now justify the reasons for defining polarizations and the polarized Hilbert space and operators as such.

First off, we analyze the intuition behind the three conditions needed to consider a distribution $P \hookrightarrow T^\mathbb{C}M$ to be a polarization:
\begin{itemize}
\item The requirement that $P$ be \emph{Lagrangian} comes from the need to find a ``representation'' or ``choice'' of configuration space inside of the phase space given by the symplectic manifold. Since the quantization procedure acts on symplectic manifolds, the question of a choice of configuration space is implicitly ignored; Lagrangian distributions, which are necessarily rank $n$ when $dim(M) = 2n$, select a configuration space. The choice, however, is usually not unique\footnote{(Note that ${2n \choose n} > 1$ for $n \geq 1$.)}.

\item The \emph{involutivity} of $P$ comes frome the Frobenius theorem, which states that involutivity is equivalent to integrability (see \cite{lee}, pp. 494-505). Recall that an integrable distribution is %%%%%%%%%%%

\item Define the distribution $D = P \cap \overline{P} \cap TM$; we require that $dim(D)$ be constant in our definition, though it is an excluded condition in other treatments. Perhaps the least obvious condition, the constant rank of $D$ guarantees that %%%%%%%%%%%%%%%
%JUSTIFY...
%intuition: cutting down on the space of states to retain physicality; i.e., making sure the hilbert space is irreducible
%polarizations correspond to representations: see examples in next section - bargmann-fock & schrodinger representations
\end{itemize}
















\message{ !name(polarizations.tex) !offset(44) }

\end{document}