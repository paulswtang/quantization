%-----PREAMBLE-----
\documentclass{tufte-handout}

%use tikz-cd for comm diagrams:
%\usepackage{tikz-cd}
%use amsthm for thm-like environments:
\usepackage{amsthm}
%use thmtools to style new thm envns:
\usepackage{thmtools}

%define thrm environment
\newtheorem{thrm}{Theorem}

%define example environment
\newtheorem{example}{Example}

%define proof-sketch environment
\renewcommand{\qedsymbol}{$\triangle$}

%define definition environment
\newtheorem{defn}{Definition}

%define shorthand for exterior derivative:
\def\d{\mathrm{d}}

\title{Polarizations and Quantization}


%-----DOCUMENT BODY-----
\begin{document}

In this chapter we will introduce and justify the notion of a \emph{polarization} on a symplectic manifold, as well as give some examples and showing how it solves some of the issues with pre-quantization by adjusting the prequantum Hilbert space into a polarized prequantum Hilbert space and adjusting the prequantized operators into polarized prequantum operators; note that there is one final modification to be made after the polarization step, however, to remedy a final issue with the integrability of states.

%===========================================================================================
\section{Definition and Basic Results}

We will hit the ground running with the definitions and proceed to justify it shortly after:

\begin{defn}
Let $(M^{2n},\omega)$ be a symplectic manifold. A (complex) distribution $P$ on $M$, i.e. a sub-bundle
$$
P \hookrightarrow T^{\mathbf{C}}M \equiv TM \otimes_{\mathbf{R}} \mathbf{C}
$$
of the (complexified) tangent bundle, is said to be a \emph{polarization} of $M$ if all of the following are hold:
\begin{itemize}
\item $P$ is \emph{Lagrangian}, i.e. it is an \emph{isotropic} distribution of dimension $n$; recall that a distribution is said to be isotropic if the symplectic form $\omega_x$ on each tangent space $T_xM$ vanishes on the subspace $P_x \hookrightarrow T_xM$ defined by the distribution, and that the maximal dimension of an isotropic distribution is half the dimension of the manifold.

\item $P$ is \emph{involutive}, i.e. for $X,Y$ vector fields in $P$, the commutator $[X,Y]$ is a vector field in $P$. In the literature, this is usually written in shorthand as $[P,P] \subset P$.

\item For all points $m \in M$, we have that $\mathrm{dim}(P_m \cap \overline{P}_m \cap TM)$ is constant, where the overline refers to complex conjugation.

\end{itemize}
\end{defn}

Having defined the meaning of a polarization, we can now define a way of polarizing the prequantum Hilbert space and polarizing the operators obtained through prequantization.
\begin{defn}
Given a symplectic manifold $(M^{2n},\Omega)$ with prequantum line bundle $(L \to M,\nabla)$ and prequantum Hilbert space\footnote{(Or more accurately, the metric completion of this space.)}
$$
\mathcal{H}^{preq} := \{\psi \in \Gamma(L) | \langle\psi,\psi\rangle < \infty \},
$$
we define the polarized Hilbert space $\mathcal{H}^{P}$ for $P$ a choice of polarization on $M$ as the Hilbert space of sections in $\mathcal{H}^{preq}$ that \emph{respect the polarization}, i.e.
$$
\nabla_X \psi = 0 \qquad \forall X \in \mathcal{X}(P),
$$
where $X$ is a vector field generated by $P$, i.e. the space of sections that remain fixed by parallel transport along any $X$ in the polarization.

Further, we say that an observable $f \in C^\infty(M)$ is \emph{polarizable}, or \emph{``respects the polarization''}, under the polarization $P$ iff $X_f$ the associated Hamiltonian vector field satisfies\footnote{Here, this is shorthand for the fact that $X_f$ keeps $P$ fixed.}
$$
[X_f,P] = P.
$$
When $f$ is polarizable, its corresponding \emph{polarized operator} $O_f: \mathcal{H}^{P} \to \mathcal{H}^{P}$ is the same\footnote{In other words, the primary purpose of a choice of polarization is to restrict the class of possible classical observables that may be meaningfully quantized --- it does not change the definition of the quantization of a classical observable.}
 as the prequantum operator $O^{preq}_f$.

\end{defn}

We remark that the condition placed upon the smooth functions on $M$ is necessary as not all observables respect the polarization. For instance, given a polarization $P$ on $M$ such that $\overline{P} = P$, we can find coordinates $x_1,\ldots,x_n,y_1,\ldots,y_n$ such that
$$
P = span(\{\frac{\partial}{\partial x_i}_{i=1}^n\});
$$
then one can verify that the oberservables $y_1(p),\ldots,y_n(p) \in C^\infty(M)$ fail to respect the polarization.
% see http://www.math.toronto.edu/karshon/grad/2006-07/polarizations.pdf for reference

%===========================================================================================
\subsection{Justification and Intuition.}
We now justify the reasons for defining polarizations and the polarized Hilbert space and operators as such.

First off, we analyze the intuition behind the three conditions needed to consider a distribution $P \hookrightarrow T^\mathbb{C}M$ to be a polarization:
\begin{itemize}
\item The requirement that $P$ be \emph{Lagrangian} comes from the need to find a ``representation'' or ``choice'' of configuration space inside of the phase space given by the symplectic manifold. Since the quantization procedure acts on symplectic manifolds, the question of a choice of configuration space is implicitly ignored; Lagrangian distributions, which are necessarily rank $n$ when $dim(M) = 2n$, select a configuration space. The choice, however, is usually not unique\footnote{(Note that ${2n \choose n} > 1$ for $n \geq 1$.)}.

\item The \emph{involutivity} of $P$ comes frome the Frobenius theorem, which states that involutivity is equivalent to integrability (see \cite{lee}, pp. 494-505). Recall that an integrable distribution is sort of like an index (or \emph{foliation}) of immersed manifolds:
\begin{fullwidth}
\begin{defn}[Integral Manifolds]
Let $D \subset TM$ be a distribution. We call an immersed submanifold
$$
N \hookrightarrow M
$$
an \emph{integral manifold} of $D$ if
$$
\forall p \in N \; T_p N = D_p.
$$
\end{defn}
\begin{defn}[Integrable Distribution]
A distribution $D \subset TM$ is an \emph{integrable} distribution if every point $m \in M$ lies in some integral manifold $N$ of $D$.
\end{defn}
\end{fullwidth}
Intuitively, an integrable distribution gives us a way of characterizing a collection of ``parallel'' immersed manifolds. As an illustrative example, consider $M = \mathbf{R}^4$. We have a distribution spanned by the vector fields $\partial / \partial x^1$ and $\partial / \partial x^2$; the 2-dimensional affine subspaces of $\mathbf{R}^4$ parallel to the plane spanned by these vectors are the integral manifolds of this distribution.

\item Define the distribution $D = P \cap \overline{P} \cap TM$; we require that $dim(D)$ be constant in our definition, though it is an excluded condition in other treatments. Perhaps the least obvious condition, the constant rank of $D$ guarantees that $D$ is a valid tangent distribution.

This is important, for example, when it comes to real polarizations. Suppose $P$ is a real polarization, so that $P = \overline{P}$. Then As long as $P \cap \overline{P} \cap TM$ has constant dimension, it also forms a Lagrangian distribution on $M$. Conversely, the complexification $D^{\mathbb{C}}$ of any LaGrangian distribution $D$ on $M$ gives us a real polarization.
\end{itemize}

Finally, we remark that polarizations may be thought of as ``representations'' (in a physical sense) of a given physical system; just as there aren't always unique choices of polarizations for a given symplectic manifold, there are also often many ways of representing a physical system. In the next section, we illustrate this with a few examples.

%===========================================================================================
\section{Examples: Real and K\"{a}hler Polarizations.}
In this section we provide a few examples to see what typical polarizations might look like, with specific emphasis on two very special classes of polarizations: \textit{real} polarizations and \textit{K\"{a}hler} polarizations. First, some definitions:
\begin{fullwidth}
\begin{defn}[Real and K\"{a}hler and Polarizations]
A polarization $P$ on a symplectic manifold $(M,\omega)$ is said to be:
\begin{itemize}
\item \emph{real} if $P = \overline{P}$;
\item \emph{pseudo-K\"{a}hler} if $P \cap \overline{P} = 0$.
\end{itemize}
Suppose we define a Hermitian form
$$
h^P(u,v) = i\omega(u,\overline{v})
$$
on $P$. If $P$ is pseudo-K\"{a}hler, then $ker(h^P)$ is non-degenerate; if $h^P$ is positive-definite on $P$ then we say that $P$ is a K\"{a}hler polarization.
\end{defn}
\end{fullwidth}
Recall that a K\"{a}hler manifold $(M,\omega,J)$ is a manifold $M$ with a symplectic form $\omega$ compatible with the complex structure $J$; the following theorem states that K\"{a}hler polarizations and K\"{a}hler manifolds give rise to each other.
\begin{fullwidth}
\begin{thrm}
Let $(M,\omega,J)$ be a K\"{a}hler manifold, and let $T^{\mathbb{C}}M$ be its complexified tangent bundle. Then the two complex distributions
$$
T_{(1,0)} = \{ v \in T^{\mathbb{C}_p} M | J(v) = iv \},\;\; T_{(0,1)} = \{ v \in T^{\mathbb{C}_p} M | J(v) = -iv \}
$$
are K\"{a}hler polarizations.

Conversely, if a K\"{a}hler polarization exists on a symplectic manifold $(M,\omega)$, then there exists a complex structure $J$ compatible with $\omega$ such that $(M,\omega,J)$ has the structure of a K\"{a}hler manifold.
\end{thrm}
\end{fullwidth}

We now present two different polarizations of the same manifold, and show how the choice of a real polarization versus a K\"{a}hler polarization can change the representation of the system. In our case, we have a cotangent bundle of a smooth manifold as our phase space; choosing certain real polarizations gives us the Schr\"{o}dinger and momentum representations, whereas a particular K\"{a}hler polarization gives us the holomorphic, or Bargmann-Fock, representation. The following examples are taken from Shaw, and the symplectic manifold $M$ is set as $T^*Q$ some cotangent bundle in both cases.
\subsection{Schr\"{o}dinger and Momentum Representations.}
\begin{itemize}
\item \emph{Manifold} --- Choose a cotangent bundle $M = T^*Q$, where $Q$ is some smooth manifold.
\item \emph{Basis} --- Choose the \emph{canonical basis} $\{q_i, p_j\}$.
\item \emph{Standard Symplectic Form} --- With the canonical basis, the standard symplectic form is written as
$$
\omega = \sum (dq_i \wedge dp_i).
$$
\item \emph{Polarization} --- Choose the polarization spanned by the momenta $\{\frac{\partial}{partial p_j}\}_j$. Then a section $\phi: M \to \mathbb{C}$ is polarized if
$$
\frac{\partial \phi}{\partial p_j} = 0 \; \forall j,
$$
or equivalently, is constant in the $p_j$ dimensions. This is the \emph{Schr\"{o}dinger} representation, matching up with physical intuition from basic quantum mechanics.

(If we had chosen the $\frac{\partial}{\partial q_i}_i$ to span our polarization, we would have had the \emph{momentum} representation.)
\end{itemize}

\subsection{Bargmann-Fock Representation.}
\begin{itemize}
\item \emph{Manifold} --- Choose again the cotangent bundle $M = T^*Q$, where $Q$ is some smooth manifold.
\item \emph{Basis} --- This time, choose the basis $z_j, \overline{z}_j$, where
$$
z_j = p_j + iq_j.
$$
\item \emph{Standard Symplectic Form} --- With the above basis, the standard symplectic form (as in the previous example) becomes
$$
\omega = \frac{1}{2}d\overline{z}_j \wedge dz_j,
$$
where summation notation is in effect.
\item \emph{Complex Structure} --- We also have a complex structure this time, viewing $M$ as a K\"{a}hler manifold, given by
$$
J z_j = i z_j, \; J \overline{z}_j = -i \overline{z}_j.
$$
\item \emph{Polarization} --- The K\"{a}hler polarization corresponding to $J$ is the polarization $P$ spanned by $\{\frac{\partial}{\partial \overline{z}_j}\}_j$. This means that polarized sections $\psi$ are those such that
$$
\frac{\partial \psi}{\partial \overline{z}_j} = 0, \; \forall j.
$$
This is the \emph{holomorphic}, or \emph{Bargmann-Fock} representation of $(M,\omega,J)$.
\end{itemize}


%===========================================================================================
\section{Remarks on Polarizations and the Polarized Hilbert Space.}
We now make some remarks on interesting features of polarizations on a symplectic manifold.

\subsection{Canonical Polarizations}
A natural question to ask is whether there are situations where a choice of polarizations is either canonical in some sense or otherwise unique; a further question might be whether there are categories where we have a class of functors indexed by choices of polarization.

\textbf{To be completed.}
%functoriality indexed by polarizations? functoriality somehow dependent on choice of polarization?

\subsection{BKS Kernels}
Another natural question is whether there are algebraic ways of comparing two different representations corresponding to two different polarizations on the same manifold.

\textbf{To be completed.}
%BKS kernels to compare differences between representations

%===========================================================================================
\section{History and References}

For an introduction to the theory of polarizations in the context of geometric quantization, see % http://www.math.toronto.edu/karshon/grad/2006-07/polarizations.pdf

\end{document}