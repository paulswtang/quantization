In this chapter we will introduce and justify the notion of a \emph{polarization} on a symplectic manifold, as well as give some examples and showing how it solves some of the issues with pre-quantization by adjusting the prequantum Hilbert space into a polarized prequantum Hilbert space and adjusting the prequantized operators into polarized prequantum operators; note that there is one final modification to be made after the polarization step, however, to remedy a final issue with the integrability of states.

\section{Definition and Basic Results}

We will hit the ground running with the definitions and proceed to justify it shortly after:

\begin{defn}
Let $(M^{2n},\omega)$ be a symplectic manifold. A (complex) distribution $P$ on $M$, i.e. a sub-bundle
$$
P \hookrightarrow T^{\mathbb{C}}M \equiv TM \otimes_\mathbb{R} \mathbb{C}
$$
of the (complexified) tangent bundle, is said to be a \emph{polarization} of $M$ if all of the following are hold:
\begin{itemize}
\item $P$ is \emph{Lagrangian}, i.e. it is an \emph{isotropic} distribution of dimension $n$; recall that a distribution is said to be isotropic if the symplectic form $\omega_x$ on each tangent space $T_xM$ vanishes on the subspace $P_x \hookrightarrow T_xM$ defined by the distribution, and that the maximal dimension of an isotropic distribution is half the dimension of the manifold.

\item $P$ is \emph{involutive}, i.e. for $X,Y$ vector fields in $P$, the commutator $[X,Y]$ is a vector field in $P$. In the literature, this is usually written in shorthand as $[P,P] \subset P$.

\item For all points $m \in M$, we have that $\mathrm{dim}(P_m \cap \overline{P}_m \cap TM)$ is constant, where the overline refers to complex conjugation.

\end{itemize}
\end{defn}

Having defined the meaning of a polarization, we can now define a way of polarizing the prequantum Hilbert space and polarizing the operators obtained through prequantization.
\begin{defn}
Given a symplectic manifold $(M^{2n},\Omega)$ with prequantum line bundle $(L \to M,\nabla)$ and prequantum Hilbert space 
$$
\mathcal{H}^{preq} := \{\psi \in \Gamma(L) | \langle\psi,\psi\rangle < \infty \},	%technically, the completion of this space
$$
we define the polarized Hilbert space $\mathcal{H}^{P}$ for $P$ a choice of polarization on $M$ as the Hilbert space of sections in $\mathcal{H}^{preq}$ that \emph{respect the polarization}, i.e.
$$
\nabla_X \psi = 0 \forall X \in \Chi(P),
$$
where $X$ is a vector field generated by $P$, i.e. the space of sections that remain fixed by parallel transport along any $X$ in the polarization. %%%%%%%%STUFF HERE; FACT-CHECK.

Further, we say that an observable $f \in C^\infty(M)$ is \emph{polarizable}, or \emph{``respects the polarization''}, under the polarization $P$ iff $X_f$ the associated Hamiltonian vector field satisfies
$$
[X_f,P] = P.
$$
When $f$ is polarizable, its corresponding \emph{polarized operator} $O_f: \mathcal{H}^{P} \to \mathcal{H}^{P}$ is the same\footnote{In other words, the primary purpose of a choice of polarization is to restrict the class of possible classical observables that may be meaningfully quantized --- it does not change the definition of the quantization of a classical observable.}
 as the prequantum operator $O^{preq}_f$.

\end{defn}

We remark that the condition placed upon the smooth functions on $M$ is necessary as not all observables respect the polarization, since %SOME REASON DUE TO REDUCIBILITY ISSUES


\subsection{Justification and Intuition.}
We now justify the reasons for defining polarizations and the polarized Hilbert space and operators as such.

First off, we analyze the intuition behind the three conditions needed to consider a distribution $P \hookrightarrow T^\mathbb{C}M$ to be a polarization:
\begin{itemize}
\item The requirement that $P$ be \emph{Lagrangian} comes from the need to find a ``representation'' or ``choice'' of configuration space inside of the phase space given by the symplectic manifold. Since the quantization procedure acts on symplectic manifolds, the question of a choice of configuration space is implicitly ignored; Lagrangian distributions, which are necessarily rank $n$ when $dim(M) = 2n$, select a configuration space. The choice, however, is usually not unique. %%%%%%%%%

\item The \emph{involutivity} of $P$ comes frome the Frobenius theorem, which states that involutivity is equivalent to integrability (see \cite{lee}, pp. 494-505). Recall that an integrable distribution is %%%%%%%%%%%

\item Define the distribution $D = P \cap \overline{P} \cap TM$; we require that $dim(D)$ be constant in our definition, though it is an excluded condition in other treatments. Perhaps the least obvious condition, the constant rank of $D$ guarantees that %%%%%%%%%%%%%%%
%JUSTIFY...
%intuition: cutting down on the space of states to retain physicality; i.e., making sure the hilbert space is irreducible
%polarizations correspond to representations: see examples in next section - bargmann-fock & schrodinger representations

















\section{Examples: Real and K\"{a}hler Polarizations.}
In this section we give a few examples to see what average, ``run-of-the-mill'' polarizations look like, as well as shine a light on two very special classes of polarizations: \textit{real} polarizations and \textit{K\"{a}hler} polarizations.

%kahler and real polarizations; EXAMPLES; define the notion of a matrix group imposing a structure on a manifold

We now present two different polarizations of the same manifold, and show how the choice of a real polarization versus a K\"{a}hler polarization changes the representation. In our case, we have a cotangent bundle of a smooth manifold as our phase space; choosing certain real polarizations gives us the Schr\"{o}dinger and momentum representations, whereas a particular K\"{a}hler polarization gives us the holomorphic, or Bargmann-Fock, representation.
	\subsection{Schr\"{o}dinger and Momentum Representations.}
	We first present the two real polarizations on the cotangent bundle. %%%

	\subsection{Bargmann-Fock Representation.}
	Now we show an example of a K\"{a}hler polarization. %%%

	\subsection{Bargmann Representation of the Harmonic Oscillator.}
	As a bonus, we present an example with the Harmonic Oscillator %%%
















\section{Remarks on Polarizations and the Polarized Hilbert Space.}
We now make some remarks on interesting features of polarizations on a symplectic manifold.

\subsection{Canonical Polarizations}
A natural question to ask is whether there are situations where a choice of polarizations is either canonical in some sense or otherwise unique; a further question might be whether there are categories where we have a class of functors indexed by choices of polarization.
%functoriality indexed by polarizations? functoriality somehow dependent on choice of polarization?

\subsection{BKS Kernels}
Another natural question is whether there are algebraic ways of comparing two different representations corresponding to two different polarizations on the same manifold.
%BKS kernels to compare differences between representations