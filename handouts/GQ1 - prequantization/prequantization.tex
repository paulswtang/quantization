\section{Definition and Basic Results}
The first step of the process is prequantization --- in its simplest formulation, a symplectic manifold is prequantized into a Hilbert space given by the sections of a complex line bundle built from the manifold which possesses a connection\footnote{That is, a \emph{Koszul} connection, as the line bundle is viewed as a rank one vector bundle.} on the bundle satisfying a curvature condition. We have the following definition:
\begin{defn}
Let $(M^{2n},\omega)$ be a symplectic manifold. Then a complex line bundle
$$
\pi: L \to M
$$
paired with a (Koszul) connection
$$
\nabla: \Gamma(L) \to \gamma(L \otimes T^*M)
$$
is called a \emph{prequantum line bundle} for $(M,\omega)$ if the following curvature condition holds:
$$
\omega = \mathrm{curv}(\nabla),
$$
i.e. if the curvature 2-form and the symplectic form are equivalent.

If such a line bundle and connection exist for $(M,\omega)$, the symplectic manifold is said to be \emph{prequantizable}.
\end{defn}
In the definition above, $\Gamma(L)$ refers to the space of smooth sections, i.e. smooth functions $M \to L$ that are right inverses of $\pi$:
$$
\pi \circ \sigma = \iota_M.
$$

We will see later that the curvature condition induces a morphism of Lie algebras between the classical and quantum observables.
%%%%%%%%%%%%REMARK ON THE INTUITION BEHIND THE CURVATURE CONDITION

For further details on the curvature form, confer the classic text of Kobayashi and Nomizu \cite{kobayashinomizu}.

We note that a symplectic manifold is prequantizable if and only if the symplectic form satisfies a cohomological property:
\begin{thrm}
A symplectic manifold $(M,\omega)$ is prequantizable if and only if
$$
\biggl[\omega\biggr] \in H_{dR}^2(M,\mathbb{Z}),
$$
i.e. the second de Rham cohomology class of the symplectic form is integral.

Equivalently, $(M,\omega)$ is prequantizable iff the period of $\omega$ is integral for each integer 2-cocycle, i.e.
$$
\int_S \omega \in \mathbb{Z}
$$
for each closed $2$-cochain $S$ (i.e. $\del S = 0$) with integer coefficients.
%%%%%%%%%CHECK THIS EQUIVALENCE
\end{thrm}
For historical reasons that we will not get into, this cohomological condition is called the \emph{Bohr-Sommerfeld condition}, and the above theorem is due to A. Weil \cite{weil}.

From the existence of a prequantum line bundle, we can define a Hilbert space on the space of sections and define the prequantized observables corresponding to each $f \in C^\infty(M,\mathbf{R})}$:
\begin{defn}
Let $(M,\omega)$ be a prequantizable symplectic manifold with prequantum line bundle given by
$$
(\pi: L \to M, \nabla: \Gamma(L) \to \Gamma(L \otimes T^*M));
$$
then the \emph{prequantum Hilbert space} $\mathcal{H}_{\mbox{preq}}$ is given by (the completion of) the space of square-integrable sections of $L$, i.e.
$$
\mathcal{H}_{\mbox{preq}} := \{\phi | \langle\phi,\phi\rangle < \infty \},
$$
where the inner product is given by
$$
\langle \phi, \psi \rangle := \int_M \langle \phi,\psi \rangle_h \omega^n
$$
where $\langle \cdot,\cdot \rangle_h$ is a hermitian inner product compatible with the connection, i.e. %%%%%%%%%FIGURE OUT/CLARIFY

For a classical observable $f \in C^\infty(M)$, the associated prequantized observable $O_{\mbox{preq}}(f)$ is defined as
$$
O_{\mbox{preq}}(f) = i\hbar\nabla_{X_f} + f,  %FIX THIS & FIGURE OUT
$$
where $X_f \in \mathcal{X}(M)$ is the associated symplectic vector field to $f$, i.e. the unique $X_f \in \mathcal{X}(M)$ such that
$$
\omega(X_f,Y) \equiv \d f (Y) = Yf.
$$
\end{defn}

%remarks on this definition
%We can confirm that this is a Hilbert space based on the properties of the integral. Intuitively, the above definition stems from the following train of thought: a first, na\"{i}ve attempt at constructing a complex Hilbert space from a given symplectic manifold might be to consider smooth, square-integrable functions $M \to \mathbb{C}$, i.e. considering the space $L^2(M,\mathbb{C})$ where the inner product is defined by an integral with respect to a certain volume form by mimicking the case of a cotangent bundle; one is then faced with the problem of 
















\section{Examples}
%include example of why prequantization map is un-physical: particle in a box, for example
We now give three examples: a basic example of prequantization, and two examples displaying the inherent un-physicality of the prequantized Hilbert space and the prequantized operators.

\begin{itemize}
\item \emph{(Cotangent Bundles.)} We first consider the prequantization of a cotangent bundle $T^*Q$ where $Q$ is a smooth manifold of dimension $n$. The cotangent bundle has a natural global symplectic 2-form given by
$$
\omega = \sum_{i=1}^n \d q_i \wedge \d p_i
$$
where $(q_i,p_i)$ are \emph{canonical coordinates}; note also that $\omega = -\d\theta$ for
$$
\theta = \sum_{i=1}^n p_i \d q_i,
$$
the canonical 1-form on $T^*Q$.

We note that the cohomology class $[\omega] = 0 \in \mathbb{Z}$ since we have that $\omega$ is exact ($\d(-\theta) = \omega$).

Hence the cotangent bundle is prequantizable; since $\omega$ is equivalently the curvature form, we have a flat connection\footnote{See theorems 9.1 and 9.2 of \cite{kobayashinomizu}, for example, for a proof of the claim that a connection is flat iff the bundle is trivial.} $\nabla$ and it is thus sufficient to take $T^*Q \times \mathbb{C}$ as the prequantum line bundle.

We obtain a prequantum Hilbert space of $\mathcal{H} = \mathcal{C}(T^*Q)$ with metric given by %%%%%%%%%%%%

The quantized operator corresponding to $f \in C^\infty(T^*Q)$ is then given by
$$
%%%%%%%%%%%%%%%%%%%%%%%%%%%%%.
$$

\item \emph{($\mathbb{R}^3$.)} As a specific example of the above, we consider $Q = \mathbb{R}^3$. According to the above, we have the symplectic manifold $T^*\mathbb{R}^3 \equiv \mathbb{R}^6$, with the hermitian line bundle given by $\mathbb{R}^6 \times \mathbb{C}$; we note that this is unique as Euclidean space is simply connected (i.e. all loops in the fundamental group are contractible), and the associated connection $\nabla$ is given by %%%%

Choose the Hamiltonian function $H$ to be %%%%



\item \emph{(Harmonic Oscillator.)} In the case of the $n$-dimensional harmonic oscillator, we have the phase space/symplectic manifold $(M,\Omega)$ given by
$$
M = \{(q^j,p_j) \in \mathbf{R}^{2n}\}, \Omega = \sum_j dq^j \wedge dp_j,
$$
with the Hamiltonian $H$ given by
$$
H(q,p) = \frac{1}{2}\sum_j ((q^j)^2 + (p_j)^2).
$$
We see that Born-Sommerfeld condition is satisfied, since we have $[\Omega] = 0 \in \mathbf{Z}$, and the line bundle is trivial and unique since $M$ is simply connected, with $L = M \times \mathbf{C}$ and the associated connection \nabla given by
$$
\nabla_X \psi = X \psi + (2\pi i)\omega(X)\psi = %%%%%%
$$
where $\omega$ is the pullback given by %%%%%%

Now let us try to apply the prequantization map to the energy, i.e. $H$; we have that
$$
O_H = -i\hbar\sum_j\biggl[p_j\frac{\partial}{\partial q^j} - q^j\frac{\partial}{\partial p_j} \biggr];
$$
further, we have that
$$
O_H(\psi) = i\hbar\{H,\psi\},
$$
where the section $\psi$ is interpreted as a function $M \to \mathbf{C}$.

There's an issue here, however; the operator $O_H$ and the classical Hamiltonian $H$ have the same spectrum, and hence $O_H$ has a continuous spectrum. But we know from physical experience that quantum operators must have discretized spectra.
%%%%% talk about coherent states here: https://en.wikipedia.org/wiki/Coherent_states
\end{itemize}













\section{Structural Properties}

As we mentioned before, the prequantization map satisfies ``fairly decent'' uniqueness properties, in the sense that the first de Rham cohomology group $H^1(M,\mathbb{T})$ parametrizes the possible prequantum line bundles one may have when given a symplectic manifold.
%prequantization has nice structure, though: show that it is functorial? --> probably not given the cohomological condition...

In particular, we have that the following:
\begin{thrm}
Let $(M,\omega)$ be a prequantizable symplectic manifold. Then the prequantum line bundle $(L\to M, \nabla)$ is unique if and only if the first de Rham cohomology group $H^1(M,\mathbb{T})$ is trivial, where $\mathbb{T} \subset \mathbb{C}$ is the group of complex numbers with unit modulus.%%%HOW TO SHOW THIS? SNYATYCKI PP53-56, WOODHOUSE P160.
\end{thrm}
In other words, the prequantum line bundle is unique when $M$ is simply connected.

In the first section of this chapter, we lied by omission --- the prequantum Hilbert space is defined not only by a complex bundle $L \to M$ and a connection $\nabla$, but also by a choice of hermitian metric $\langle \cdot,\cdot \rangle_h$ (or just $h: M \times M \to \mathbb{C}$). Towards this, we have the following:
\begin{thrm}
%%%%CLASSIFCATION OF HERMITIAN METRICS ON SYMPL. MANIFOLDS
\end{thrm}

Further, as implied earlier, we have a morphism of Lie algebras between the classical and quantum observables:
\begin{thrm}
%%%%%%MORPHISM OF LIE ALGEBRAS INDUCED BY CURVATURE CONDITION
\end{thrm}