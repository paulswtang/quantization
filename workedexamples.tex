In this chapter we present three fully worked examples of the overall geometric quantization process. The first example is a continuation of the quantization process of the harmonic oscillator, as in previous examples, and the second example is of flat Euclidean space. The last example is of the 2-sphere, which possesses not just a symplectic structure, but a \emph{K\"{a}hler} structure --- the symplectic structure exists and is compatible with extant Riemannian and complex structures.

\section{Harmonic Oscillator}
%simms quantization of harmonic oscillator
In the case of the harmonic oscillator, we have the phase space given by the manifold
$$
M = \{ (p,q) \in \mathbb{R}^{2n}: \}						%FIGURE THIS OUT
$$
with the symplectic form given by
$$
\omega = \frac{1}{2}\sum_{i=1}^n \d p^i \wedge \d q^i.		%CHECK THIS
$$

\subsection{Prequantization}

\subsection{Polarizations}

\subsection{Correction}







%%%%%%%%%%%%%%%%%%%%%%%%%%%%%%%%%%%%%%%%%%%%%%%%%%%%%%%%%%%
\section{Cotangent Bundles}
As a more general example, we will quantize a cotangent bundle with the natural symplectic form. Let $Q^n$ be some smooth manifold with cotangent bundle $T^*Q$ equipped with canonical symplectic form
$$
\omega = \sum_i \d q_i \wedge \d p_i
$$
arising from the tautological 1-form
$$
\theta = \sum_i q_i \d p_i.
$$

We can confirm that this is a symplectic manifold: we have that %%%%%%%

\subsection{Prequantization}

We have that the manifold is prequantizable since the cohomology class of $\omega$ is integral:
$$
[\omega] = 
$$

\subsection{Polarizations}

\subsection{Correction}








%%%%%%%%%%%%%%%%%%%%%%%%%%%%%%%%%%%%%%%%%%%%%%%%%%%%%%%%%%%
\section{\mathbb{S}^2}
We now present a quantization of the $2$-sphere $\mathbb{S}^2$ viewed as a K\"{a}hler manifold\footnote{Ordinarily, we would consider the general $n$-dimensional sphere for $n = 2k$, but it actually turns out that the only sphere for which a symplectic structure exists is $n=2$! This is a consequence of the fact that the second de Rham cohomology $H^2(S^n)$ is trivial for $n > 2$.}; we note that the Riemannian, complex, and symplectic structures are constructed as follows:

\begin{itemize}
\item There is a canonical embedding
$$
S^2 \hookrightarrow \mathbb{R}^3
$$
obtained in the obvious way; since 3-dimensional Euclidean space has a canonical metric
$$
\sum_{i=1}^3 \d x_i \wedge \d x_i,
$$
so does the 2-sphere, via the pullback through the embedding:
$$
%%%%%%%%%
$$

This gives us our Riemannian structure.

\item $S^2$ may be given an atlas consisting of two copies of $\mathbb{C}$ as the coordinate charts in the following manner:%%%%%%%%%%

\item Lastly, the symplectic structure of $S^2$ is given by taking the symplectic form to be the canonical volume form induced by the Riemannian structure; in other words,

$$
\omega = %%%%%%%%%%%%%%
$$

\end{itemize}

It is not difficult to confirm that these three structures are compatible: %%%%%%%%%

\subsection{Prequantization}

Of course, we first need to confirm that this manifold is in fact prequantizable by computing the integral of the symplectic form over an arbitrary $2$-cocycle:
$$
\int_S \omega = %%%%%%%%%%%%%
$$

We now consider the question of the line bundle itself; %%%%%%%%%%%%%%%%%

\subsection{Polarization}

Since our manifold is K\"{a}hler, there is a canonical polarization given by%%%%%%%%%%%%%%%

\subsection{Correction}







For further references, see the following presentations:

%http://ncatlab.org/nlab/show/geometric+quantization+of+the+2-sphere
%http://math.stackexchange.com/a/952282