\section{Comments}
This current document is a survey of two popular quantization schemes: geometric quantization and deformation quantization. An emphasis is placed on the structural, category-theoretic properties of these maps rather than their analytic components. Additionally, further generalizations and current work in these fields are discussed.

No claim of originality is made, as the dual purposes of this survey are to function as an exposition of these two techniques, and to summarize some components of contemporary work on the methods involved. The outline of the survey is as follows:
\begin{itemize}
\item Part I is an overview of geometric quantization, as well as a discussion of some modern views on the process;
\item Part II discusses algebraic quantization;
\item Part III discusses the inter-relations and connections between algebraic and geometric quantization, and quantization in general; and finally,
\item the final part is a series of appendices on pre-requisites and asides that elaborate on topics mentioned in the previous parts that were not fully explored.
\end{itemize}
 
The reader will find this document most helpful if they possess familiarity with some elements of differential geometry (symplectic geometry, fibre bundles, and connections), functional analysis (Hilbert spaces, algebras of operators, basic operator theory), representation theory and group theory (Lie groups and Lie algebras, representations) and category theory (categories, functors, natural transformations, some aspects of the theory of $n$-categories and higher category theory, and topos theory). There will be some exposition of the physical models involved, but the primary motivation is mathematical; hence, some intuition on the connections between Hamiltonian mechanics and symplectic geometry and on the connections between quantum mechanics and Hilbert spaces will be helpful.

%motivation from philosophy
Some initial remarks on the content of this survey are in order. \textit{Quantization} is generally accepted in this context to refer to a structured process that passes from the formalism of classical mechanics to the formalism of quantum mechanics. In general, \emph{there is no exact recipe} for quantization. Some reasons for this involve the existence of ``no-go'' theorems that prevent certain convenient properties from holding, and the somewhat unavoidable arbitrary choices required during quantization\footnote{Polarizations, for example, are often not unique, and there may not be a preferred choice.}. Consequently, there are many different quantization methods, each with their own motivation --- in the current survey, we discuss two of them: geometric quantization (which attempts to preserve geometric structure and hence avoids reliance on coordinates), and deformation quantization (which attempts to find a ``deformation'' of the algebra of observables from the commutative classical context to the non-commutative quantum context). It must be said that these two schemes correspond to quantization with a focus on physical states versus quantization with a focus on observables, not unlike the distinction between Heisenberg's matrix mechanics and Schr\"{o}dinger's wave function.

%classical limit is not as obvious as many think
%It is a widespread belief that the classical limit is well understood, or otherwise that there are few problems going from the quantum setting to the classical one. In truth, it is not as obvious as many believe. While the classical or semi-classical limits are well understood from an analytic point of view (in that one may formally take $\hbar \to 0$ in some appropriate sense and recover certain structures), and the algebraic/deformation aspects\footnote{In this context, we refer to the idea of taking a non-commutative family of algebras indexed by a formal parameter $\hbar$ and ``deforming'' them along this parameter until we obtain commutativity.} are understood (but significantly more complicated), the preservation of geometric aspects is not very well understood. This is covered in Part III.

Special thanks go to Prakash Panangaden (School of Computer Science, McGill University) for introducing the author to the topic, and to Jesse Madnick (Department of Mathematics, Stanford University) for helpful discussions on K\"{a}hler geometry.